% Chapter Chapter 11 For Reproducible Research in R and RStudio
% Christopher Gandrud
% Created: 16/07/2012 05:45:03 pm CEST
% Updated: 29 September 2012




\chapter{Presenting with LaTeX}\label{LatexChapter}

\section{The Basics}

All commands in LaTeX start with a \texttt{\textbackslash{}}

\subsection{Editors}

As I mentioned earlier, RStudio is an fully functional LaTeX editor as well as an integrated development environment for R. Of course it is oriented towards combining R and LaTeX. If you want to create a new LaTeX document you can click {\tt{File}} \rightarrow {\tt{New}} \rightarrow {\tt{R\; Sweave}}. 

Remember from Chapter \ref{GettingStartedRKnitr} that R Sweave\index{R Sweave} files are basically LaTeX files that can include {\emph{knitr}} code chunks. You can compile R Sweave files like regular LaTeX files in RStudio even if they do not have code chunks. If you use another program to compile them you might need to change the file extension from {\tt{.Rnw}} to {\tt{.tex}}.

\subsection{The header \& the body}

All LaTeX documents require a header\index{LaTeX header}. The header goes before the body of the document and specifies what type of presentation document you are creating--an article, a book, a slideshow, and so on. LaTeX refers to these as classes\index{LaTeX class}. You also can specify what style it should be formatted in and load any extra packages you may want to use to help you format your document.\footnote{The command to load a package in LaTeX is \texttt{\textbackslash{}usepackage}. For example, if you include \texttt{\textbackslash{}usepackage\{url\}} in the header of your document you will be able to specify URL links in the body with the command \texttt{\textbackslash{}url\{SOMEURL\}}.}

The header is followed by the body of your document. You tell LaTeX where the body\index{LaTeX begin document} of your document starts by typing \texttt{\textbackslash{}begin\{document\}}. The very last line of you document is usually \texttt{\textbackslash{}end\{document\}}, indicating that your document has ended. When you open a new R Sweave file in RStudio it creates an article class document with a very simple header and body like this:

\begin{knitrout}
\definecolor{shadecolor}{rgb}{0.969, 0.969, 0.969}\color{fgcolor}\begin{kframe}
\begin{alltt}
\textbackslash{}documentclass\{article\}
\textbackslash{}begin\{document\}
\textbackslash{}end\{document\}
\end{alltt}
\end{kframe}
\end{knitrout}



\subsection{Headings}

\subsection{Footnotes \& Bibliographies}

\subsubsection{Footnotes}

Plain, non-bibliographic footnotes are easy to create in LaTeX. Simply place \texttt{\textbackslash{}footnote\{} where you would like the footnote number to apear in the text. Then type in the footnote's text and of course remember to close it with a \texttt{\}}. LaTeX does the rest, including formatting and numbering.

\subsubsection{Bibliographies}

\paragraph{Citing R Packages with BibTeX}

Researchers are pretty good about consistently citing others' articles and data. However, citing the R packages used in an analysis is very inconsistent. This is unfortunate not only because correct attribution is not being given but also because it makes reproducibility harder because it obscures important steps that were taken in the
research process. Fortunately, R actually includes the tools to quickly generate citations, including the version of the package you are using. It can also add them directly to an existing bibliography file.

You can automatically create citations for R packages using the \texttt{citation} command in \emph{base} R. For example if we want the citation information for the \texttt{Zelig} package we would simply type:

\begin{knitrout}
\definecolor{shadecolor}{rgb}{0.969, 0.969, 0.969}\color{fgcolor}\begin{kframe}
\begin{alltt}
\hlfunctioncall{citation}(\hlstring{"xtable"})
\end{alltt}
\begin{verbatim}
## 
## To cite package 'xtable' in publications use:
## 
##   David B. Dahl (2012). xtable: Export tables to
##   LaTeX or HTML. R package version 1.7-0.
##   http://CRAN.R-project.org/package=xtable
## 
## A BibTeX entry for LaTeX users is
## 
##   @Manual{,
##     title = {xtable: Export tables to LaTeX or HTML},
##     author = {David B. Dahl},
##     year = {2012},
##     note = {R package version 1.7-0},
##     url = {http://CRAN.R-project.org/package=xtable},
##   }
## 
## ATTENTION: This citation information has been
## auto-generated from the package DESCRIPTION file and
## may need manual editing, see 'help("citation")' .
\end{verbatim}
\end{kframe}
\end{knitrout}


\noindent This gives us both the plain citation as well as the BibTeX version for use in LaTeX and MultiMarkdown documents. If you only want the BibTeX version of the citation we can use the \texttt{toBibtex} command in the \emph{utils} package.

\begin{knitrout}
\definecolor{shadecolor}{rgb}{0.969, 0.969, 0.969}\color{fgcolor}\begin{kframe}
\begin{alltt}
\hlfunctioncall{toBibtex}(\hlfunctioncall{citation}(\hlstring{"xtable"}))
\end{alltt}
\begin{verbatim}
## @Manual{,
##   title = {xtable: Export tables to LaTeX or HTML},
##   author = {David B. Dahl},
##   year = {2012},
##   note = {R package version 1.7-0},
##   url = {http://CRAN.R-project.org/package=xtable},
## }
\end{verbatim}
\end{kframe}
\end{knitrout}


\noindent You can append the citation to your existing BibTeX file using the \texttt{sink} command in \emph{base} R. This command diverts our output and/or the messages to a file. For example, imagine that our existing BibTeX file is called \texttt{biblio.bib}. To add the \emph{Zelig} package citation:

\begin{knitrout}
\definecolor{shadecolor}{rgb}{0.969, 0.969, 0.969}\color{fgcolor}\begin{kframe}
\begin{alltt}
\hlcomment{# Divert output to biblio.bib}
\hlfunctioncall{sink}(file = \hlstring{"biblio.bib"}, 
     append = TRUE, type = \hlfunctioncall{c}(\hlstring{"output"})
     )      
\hlcomment{# Extract BibTeX citation     }
\hlfunctioncall{toBibtex}(\hlfunctioncall{citation}(\hlstring{"xtable"})) 
\hlfunctioncall{sink}()
\end{alltt}
\end{kframe}
\end{knitrout}


\noindent This places the citation at the end of our \texttt{biblio.bib} file. It is very important to include the argument \texttt{append = TRUE}. If you don't you will erase the existing file. The argument \texttt{type = c("output")} tells R to include only the output, not the messages.

An even faster way to add citations to a bibliography is with \texttt{write.bibtex} command in the \emph{knitcitations} package. To add the \emph{Zelig} citation to our \texttt{biblio.bib} file we only need to enter:

\begin{knitrout}
\definecolor{shadecolor}{rgb}{0.969, 0.969, 0.969}\color{fgcolor}\begin{kframe}
\begin{alltt}
\hlcomment{# Load package}
\hlfunctioncall{library}(knitcitations)
 
\hlcomment{# Write Zelig citation and}
\hlcomment{# to biblio.bib}
\hlfunctioncall{write.bibtex}(entry = \hlfunctioncall{c}(\hlstring{"xtable"}), 
              file = \hlstring{"bibliography.bib"}, append = TRUE)
\end{alltt}
\end{kframe}
\end{knitrout}


\noindent Note, you will likely only want to append the citations once. Otherwise your bibliography document will grow with redundant information every time you run this command.

The {\emph{knitr}} package can also create BibTeX bibliographies for R packages using the \texttt{write\_bib} command. To use this command you list the packages whose citation details you want to include in a specified file. The command currently does not have the ability to append the citations to an existing file, but instead writes them to a new file.

\section{Presentations with Beamer}

You can make slideshow presentations with LaTeX. FILL IN WITH INTRO

{\emph{knitr}} largely the works the same way in in LaTeX slideshows as it does in article or book class documents. There are a few differences to look out for. 

\paragraph{Slide frames}

A quick way to create each Beamer slide is to use the \texttt{frame} command:

\begin{knitrout}
\definecolor{shadecolor}{rgb}{0.969, 0.969, 0.969}\color{fgcolor}\begin{kframe}
\begin{alltt}
\textbackslash{}frame\{
\}
\end{alltt}
\end{kframe}
\end{knitrout}


\noindent If you want to include highlighted {\emph{knitr}} code chunks on your slides you should add the \texttt{fragile} option to the \texttt{frame} command.\footnote{For a detailed discussion of why you need to use the \texttt{fragile} option with the verbatim environment that {\emph{knitr}} uses to display highlighted text in LaTeX documents see this blog post by Pieter Belmans: \url{http://pbelmans.wordpress.com/2011/02/20/why-latex-beamer-needs-fragile-when-using-verbatim/}.} Here is an example:

\begin{knitrout}
\definecolor{shadecolor}{rgb}{0.969, 0.969, 0.969}\color{fgcolor}\begin{kframe}
\begin{alltt}
\textbackslash{}begin\{frame\}[fragile]
    An example fragile frame.
\textbackslash{}end\{frame\}
\end{alltt}
\end{kframe}
\end{knitrout}


\paragraph{Results}

By default {\emph{knitr}} hides the results or a code chunk. If you want to show the results in your slideshow simply set the {\tt{results}} option to {\tt{'asis'}}.
