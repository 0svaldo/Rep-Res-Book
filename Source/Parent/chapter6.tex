% Chapter Chapter 6 For Reproducible Research in R and RStudio
% Christopher Gandrud
% Created: 16/07/2012 05:45:03 pm CEST
% Updated: 14 October 2012




\chapter{Gathering Data with R}\label{DataGather}

How you gather your data directly impacts how reproducible your research will be. Of course you should try your best to document everything. Reproduction will be easier if your documentation--especially, variable descriptions and source code--makes it so that you and others can understand what you have done. If all of your data gathering steps are embedded in source code, then independent researchers (and you) can more easily regather the data. Regathering data will be easiest if your source code can tie your analysis code all the way back to the raw data--the rawer the better. Of course this may not always be possible. You may need to conduct interviews or compile information from paper based archives, for example. The best you can sometimes do is describe your data gathering process in detail. Nonetheless, R's automated data gathering capabilities for internet-based information are extensive. Learning how to take full advantage of them greatly increases reproducibility and can save you considerable time and effort.

In this chapter you'll learn how to gather quantitative data in a reproducible and, in some cases, fully replicable way. You'll start by learning how to use data gathering make files\index{make file} to organize your whole data gathering process so that it can be completely reproduced. Then you will learn the details of how to actually load data into R from various sources, both locally on your computer and via the internet. In the next chapter (Chapter \ref{DataClean}) you'll learn the details of how to clean up raw data so that it can be merged together into data frames that you can use for statistical analyses.

%%%%%%%%%%%%% Organizing data gathering
\section{Organize your data gathering: make files}

Before getting into the details of using R to gather data, lets's start by creating a plan to organize the process. Organizing your data gathering process from the beginning of a research project improves the possibility of reproducibility and can save you significant effort over the course of the project by making it easier to add and regather data later on. 

A key part of reproducible data gathering with R, like reproducible research in general is segmenting the process into discrete files that can be run by a common make file\index{make file}. Segmentation makes it easier to navigate research text and find errors in the source code. The make file's output is the data set(s) that you'll use in the statistical analyses. There are two types of source code files that the make file runs: data gathering/clean up files and merging files. Data clean up files bring raw (the rawer the better) individual data sources into R and transform them so that can be merged with data from the other sources. Some of the R tools for data clean up and merging will be covered in Chapter \ref{DataClean}. In this chapter we mostly cover the ways to bring raw data into R. Merging files are executed by the make file after it runs the data gathering/clean up files.

It's a good idea to have the source code files use very raw data as input. Your source code should avoid directly changing these raw data files. Instead changes should be output to new objects and data files. Doing this makes it easier to reconstruct the steps you took to create your data set. Also, while cleaning and merging your data you may transform it in an unintended way, for example, accidentally deleting some observations that you had wanted to keep. Having the raw data makes it easy to go back and correct your mistakes. 

In data gathering make files you usually only need one or two commands {\tt{setwd}}\index{setwd} and {\tt{source}}\index{source}. As we talked about in Chapter \ref{DirectoriesChapter}, {\tt{setwd}} simply tells R where to look for and place files. {\tt{source}} tells R to run code in an R source code file.\footnote{We use the {\tt{source}} command is used more in the Chapter \ref{StatsModel}.}  Lets see what a data make file might look like for our example project (see Figure \ref{ExampleTree}). The file paths in this example are for Unix-like systems.

\begin{knitrout}
\definecolor{shadecolor}{rgb}{0.969, 0.969, 0.969}\color{fgcolor}\begin{kframe}
\begin{alltt}
\hlcomment{################}
\hlcomment{# Example Make file}
\hlcomment{# Christopher Gandrud}
\hlcomment{# Updated 10 October 2012}
\hlcomment{################}

\hlcomment{# Set working directory}
\hlfunctioncall{setwd}(\hlstring{"/ExampleProject/Data/"})

\hlcomment{# Gather and clean up raw data files.}
\hlfunctioncall{source}(\hlstring{"/GatherSource/IndvDataGather/Gather1.R"})

\hlfunctioncall{source}(\hlstring{"/GatherSource/IndvDataGather/Gather2.R"})

\hlfunctioncall{source}(\hlstring{"/GatherSource/IndvDataGather/Gather3.R"})
    
\hlcomment{# Merge cleaned data files into object CleanedData}
\hlfunctioncall{source}(\hlstring{"GatherSource/MergeData.R"})

\hlcomment{# Save cleaned & merged Data as MainData.csv}
\hlfunctioncall{write.csv}(CleandedData, file = \hlstring{"/DataFiles/MainData.csv"})
\end{alltt}
\end{kframe}
\end{knitrout}


This code first sets the working directory. Then the make file runs three source code files to gather data from three different sources. These files gather the data and clean it so that it can be merged together. Next the make file runs a source code file that merges the data frames. Finally, it saves the output data frame {\emph{CleanData}} as a \texttt{.csv} formated file called {\emph{MainData.csv}} using the {\tt{write.csv}}\index{write.csv} command. In our example project, this would be the main file we use for statistical analysis. 

Now that we've covered the big picture, let's learn the different tools you will need to know to gather data from different types of sources.

\section{Importing locally stored data sets}

The most straightforward place to load data from is a local file, e.g. one stored on your computer. Though storing your data locally does not really encourage reproducibility, most research projects will involve loading data this way at some point. The tools you will learn for importing locally stored data files will also be important for most of the other methods further on. Let's briefly look at how to load single and multiple files locally.

\subsection{Importing a single locally stored file}

As we have seen, plain-text file based data stored on your computer can be loaded into R using the \texttt{read.table}\index{read.table} command. If you are using RStudio you can do the same thing with drop down menus. To open a plain-text data file click on \texttt{Workspace} \textrightarrow\: \texttt{Import Dataset\ldots} \textrightarrow\: \texttt{From Text File\ldots}. In the box that pops up, specify the separator, whether or not you want the first line to be treated as variable labels, and other options. This is initially easier than using \texttt{read.table}. But it is less reproducible.

If the data is not stored in plain-text format, but is instead saved by another statistical program such as SPSS, SAS, or Stata, we can import it using commands in the \emph{foreign} package\index{foreign}. For example, imagine we have a data file called \emph{Data1.dta} stored in our working directory. This file was created by the Stata\index{Stata} statistical program. To load the data into an R data frame object called \emph{StataData} simply type:

\begin{knitrout}
\definecolor{shadecolor}{rgb}{0.969, 0.969, 0.969}\color{fgcolor}\begin{kframe}
\begin{alltt}
\hlcomment{# Load library}
\hlfunctioncall{library}(foreign)

\hlcomment{# Load Stata formatted data}
StataData <- \hlfunctioncall{read.dta}(file = \hlstring{"Data1.dta"})
\end{alltt}
\end{kframe}
\end{knitrout}


As you can see, commands in the \emph{foreign} library have similar syntax to \texttt{read.table}. To see the full range of commands and file formats that the \emph{foreign} package supports use the following command:

\begin{knitrout}
\definecolor{shadecolor}{rgb}{0.969, 0.969, 0.969}\color{fgcolor}\begin{kframe}
\begin{alltt}
\hlfunctioncall{library}(help = \hlstring{"foreign"})
\end{alltt}
\end{kframe}
\end{knitrout}


If you have data stored in a spreadsheet format such as Excel's \texttt{.xlsx}, it may be best to first clean up the data in the spreadsheet program by hand and then save the file in plain-text format. When you clean up the data make sure that the first row has the variable names and that observations are in the following rows. Also, remove any extraneous information--notes, colors, and so on--that will not be part of the data frame.

To aid reproducibility, locally stored data should include careful documentation of where the data came from and how, if at all, it was transformed before it was loaded into R. Ideally the documentation would be written in a text file saved in the same directory as the raw data file. 

\subsection{Looping through multiple files}

\todo[inline]{This subsection needs to be written.}

\section{Importing data sets from the internet}

There are many ways to import data that is stored on the internet directly into R. We have to use different methods depending on where and how the data is stored. 

\subsection{Data from non-secure ({\tt{http}}) URLs}

Importing data into R that is located at a non-secure URL--ones that start with {\tt{http}}--is straightforward provided that:

\begin{itemize}
    \item the data is stored in a simple format, e.g. plain-text,
    \item the file is not embedded in a larger HTML website.
\end{itemize}

We have discussed the first issue in detail. You can determine if the data file is embedded in a website by opening the URL. If you only see the raw plain-text data, you are probably good to go.\footnote{If the data is embedded in a larger website--the way data is usually stored on the cloud version of Dropbox--you may still be able to download it into R. However, this can be difficult and varies depending on the structure of the website. So, I do not cover it in this book.}

To import the data simply include the URL as the file name in your \texttt{read.table} command. We saw earlier how to download a CSV data file from a Dropbox \emph{Public} folder with the shortened URL \texttt{http://bit.ly/PhjaPM}:

\begin{knitrout}
\definecolor{shadecolor}{rgb}{0.969, 0.969, 0.969}\color{fgcolor}\begin{kframe}
\begin{alltt}
Data <- \hlfunctioncall{read.table}(\hlstring{"http://bit.ly/PhjaPM"}, 
                    sep = \hlstring{","}, header = TRUE)
\end{alltt}
\end{kframe}
\end{knitrout}


\subsection{Data from secure ({\tt{https}}) URLs}

\noindent We have to take a few extra steps to download data from a secure URL. You can tell if the data is stored at a secure web address if it begins with \texttt{https} rather than \texttt{http}. We need the help of the \texttt{getURL}\index{getURL} command in the {\emph{RCurl}} package \cite[]{R-RCurl} and \texttt{textConnection}. The latter command is in base R. The two rules about data being stored in plain text-formats and not being embedded in a large HTML website apply to secure web addresses as well.

Let's try an example. I have data in comma-separated values format stored at a GitHub\index{GitHub} repository. The URL for the ``raw" (plain-text) version of the data is \url{https://raw.github.com/christophergandrud/Disproportionality_Data/master/Disproportionality.csv}.\footnote{To find the URL for the raw version of a file on the GitHub website simply click the \texttt{Raw} button on the right just above the file preview.} Imagine that we put the address as a character string into an object called {\emph{UrlAddress} (not shown).\footnote{See page \pageref{Objects} for how to put a character string into an object. I do not show how this was done in the book, due to space constraints.}
To download it into R we could use this code:




\begin{knitrout}
\definecolor{shadecolor}{rgb}{0.969, 0.969, 0.969}\color{fgcolor}\begin{kframe}
\begin{alltt}
\hlcomment{# Load package}
\hlfunctioncall{library}(RCurl)

\hlcomment{# Pull data from the internet}
DataUrl <- \hlfunctioncall{getURL}(UrlAddress)

\hlcomment{# Convert Data into a data frame}
Data <- \hlfunctioncall{read.table}(\hlfunctioncall{textConnection}(DataUrl), 
                    sep = \hlstring{","}, header = TRUE)
                    
\hlcomment{# Show variables in data}
\hlfunctioncall{names}(Data)
\end{alltt}
\begin{verbatim}
## [1] "country"            "year"               "disproportionality"
\end{verbatim}
\end{kframe}
\end{knitrout}


\subsection{Compressed data stored online}

Sometimes data files are large, making them difficult to store and download without compressing\index{file compression} them. There are a number of compression methods such as Zip and tar.\footnote{Tar archives are sometimes referred to as `tar balls'.} Zip files have the extension {\tt{.zip}} and tar files use extensions such as {\tt{.tar}} and {\tt{.gz}}. In most cases\footnote{Some formats that require the {\emph{foreign}} package to open are more difficult. This is because functions such as {\tt{read.dta}} for opening Stata {\tt{.dta}} files only accept file names or URLs as arguments, not connections, which you create for unzipped files.} you can download, decompress, and create data frame objects from these files directly in R. 

To do this you need to:\footnote{The description of this process is based on a Stack Overflow comment by Dirk Eddelbuettel (see {\url{http://stackoverflow.com/questions/3053833/using-r-to-download-zipped-data-file-extract-and-import-data?answertab=votes\#tab-top}}, accessed 16 July 2012.)}

\begin{itemize}
    \item create a temporary file with {\tt{tempfile}} to store the zipped file, which you will later remove with the {\tt{unlink command}} at the end,
    \item download the file with {\tt{download.file}},
    \item decompress the file with one of the {\tt{connections}} commands in base R,\footnote{To find a full list of commands type {\tt{?connections}} into the R console.}
    \item read the file with {\tt{read.table}}. 
\end{itemize}

\noindent The reason that we have to go through so many extra steps is that compressed files are more than just a single file, but can contain a number of files as well as metadata.

Let's download a compressed file called {\emph{uds\_summary.csv}} from \cite{Pemstein2010}. It is in a compressed file called {\emph{uds\_summary.csv.gz}}. The file's URL address is {\url{http://www.unified-democracy-scores.org/files/uds_summary.csv.gz}}, that I shortened\footnote{Again, I used bitly (\url{bitly.com}) to shorten the URL.} to \url{http://bit.ly/S0vxk2} because of space constraints.

\begin{knitrout}
\definecolor{shadecolor}{rgb}{0.969, 0.969, 0.969}\color{fgcolor}\begin{kframe}
\begin{alltt}
\hlcomment{# For simplicity, store the URL in an object called \hlstring{'url'}.}
url <- \hlstring{"http://bit.ly/S0vxk2"}

\hlcomment{# Create a temporary file called \hlstring{'temp'} to put the zip file into.}
temp <- \hlfunctioncall{tempfile}()

\hlcomment{# Download the compressed file into the temporary file.}
\hlfunctioncall{download.file}(url, temp)

\hlcomment{# Decompress the file and convert it into a dataframe}
\hlcomment{# class object called \hlstring{'data'}.}
Data <- \hlfunctioncall{read.csv}(\hlfunctioncall{gzfile}(temp, \hlstring{"uds_summary.csv"}))

\hlcomment{# Delete the temporary file.}
\hlfunctioncall{unlink}(temp)

\hlcomment{# Show variables in data}
\hlfunctioncall{names}(Data)
\end{alltt}
\begin{verbatim}
## [1] "country" "year"    "cowcode" "mean"    "sd"      "median"  "pct025" 
## [8] "pct975"
\end{verbatim}
\end{kframe}
\end{knitrout}


\subsection{Data APIs \& feeds}

\todo[inline]{The remainder of this chapter is incomplete.}

There are a growing number of packages that can gather data directly from a variety of internet sources and import them into R. Needless to say, this is great for reproducible research. It not only makes the data gathering process easier as you don't have to download many of Excel files and fiddle around with them before even getting the data into R, but it also makes replicating the data gathering process much more straightforward. Some examples of these packages include: 

\begin{itemize}
    \item \todo[inline]{Need to Complete list.}
    \item The \emph{openair} package, which beyond providing a number of tools for analysing air quality data also has the ability to directly gather data directly from sources such as Kings College London's London Air (\url{http://www.londonair.org.uk/}) database with the \texttt{importKCL} command.
\end{itemize}

\section{Basic web scraping}

\subsection{Gathering and parsing text from the web}

\subsection{Scraping tables}



