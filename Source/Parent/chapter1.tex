% Chapter Chapter 1 For Reproducible Research in R and RStudio
% Christopher Gandrud
% Created: 16/07/2012 05:45:03 pm CEST
% Updated: 27 December 2012




\chapter{Introducing Reproducible Research}\label{Intro}

Research is often presented in very abridged packages: slideshows, journal articles, books, or maybe even websites. These presentation documents announce a project's findings and try to convince us that the results are correct \cite[]{Mesirov2010}. It's important to remember that these documents are not the research. Especially in the computational and statistical sciences, these documents are the ``advertising". The research is the ``full software environment, code, and data that produced the results" \cite[385]{Buckheit1995,Donoho2010}. When we separate the research from its advertisement we are making it difficult for others to verify the findings by reproducing them. 

This book gives you the tools to dynamically combine your research with the presentation of your findings. The first tool is a workflow for reproducible research that weaves the principles of reproducibility throughout your entire research project, from data gathering to the statistical analysis, and the presentation of results. You will also learn how to use a number of computer tools that make this workflow possible. These tools include:

\begin{itemize}
    \item the R statistical language that will allow you to gather data and analyze it,
    \item the LaTeX and Markdown markup languages that you can use to create documents--slideshows, articles, books, and webpages--to present your findings,
    \item the {\emph{knitr}} package and other tie commands, that dynamically tie your data gathering, analysis, and presentation documents together so that they can be easily reproduced,
    \item RStudio, a program that brings all of these tools together in one place.
\end{itemize}

%%%%%%%%%%%%%% What is reproducible research? %%%%%%%%%%%%%

\section{What is reproducible research?}

Research results are replicable if there is sufficient information available for independent researchers to make the same findings using the same procedures \cite[444]{King1995}. For research that relies on experiments, this can mean a researcher not involved in the original research being able to rerun the experiment and validate that the new results match the original ones. In computational and quantitative empirical sciences results are replicable if independent researchers can recreate findings by following the procedures originally used to gather the data and run the computer code. Of course it is sometimes difficult to replicate the original data set because of limited resources.\footnote{In this book we will actually aim for replicable research, even if we don't always achieve it. New technologies make it possible to easily replicate some kinds of data sets, especially if the original data is available over the internet.} So as a next-best standard we can aim for ``really reproducible research" \cite[1226]{Peng2011}.\footnote{The idea of really reproducible computational research was originally thought of and implemented by Jon Claerbout\index{Jon Claerbout} and the Stanford Exploration Project beginning in the 1980s and early 1990s \cite[]{Fomel2009,Donoho2009}. Further seminal advances were made by Jonathan B. Buckheit and David L. Donoho who created the Wavelab library of MatLab\index{MatLab} routines for their research on wavelets in the mid-1990s \cite[]{Buckheit1995}.} In computational sciences\footnote{Reproducibility is important for both quantitative and qualitative research \cite[]{King1994}. Nonetheless, we will focus mainly on on methods for reproducibility in quantitative computational research.} this means:

\begin{quote}
    the data and code used to make a finding are available and they are sufficient for an independent researcher to recreate the finding.
\end{quote} 

In practice, research needs to be {\emph{easy}} for independent researchers to reproduce \cite[]{Ball2012}. If a study is difficult to reproduce it's more likely that no one will reproduce it. If someone does attempt to reproduce this research, it will be difficult for them to tell if any errors they find were in the original research or problems they introduced during the reproduction. In this book you will learn how to avoid these problems. 

In particular you will learn tools for dynamically ``{\emph{knitting}}"\index{knit}\footnote{Much of the reproducible computational research and literate programming literatures have traditionally used the term ``weave"\index{weave} to describe the process of combining source code and presentation documents \cite[see][101]{Knuth1992}. In the R community weave is usually used to describe the combination of source code and LaTeX documents. The term ``knit" reflects the vocabulary of the {\emph{knitr}} R package\index{knitr} (knit + R). It is used more generally to describe weaving with a variety of markup languages. Because of this, I use the term knit rather than weave in this book.} the data and the source code together with your presentation documents. Combined with well organized source files and clearly and completely commented code, independent researchers will be able to understand how you obtained your results. This will make your computational research easily reproducible.

%%%%%%%%%%%%%% Why should research be reproducible? %%%%%%%%%%%%%

\section{Why should research be reproducible?}

Reproducibility research is one of the main components of science. If that's not enough reason for you to make your research reproducible, consider that the tools of reproducible research also have direct benefits for you as a researcher. 

\subsection{For Science}

Replicability has been a key part of scientific enquiry from perhaps the 1200s \cite[]{Bacon1267,Nosek2012}. It has even been called the ``demarcation between science and non-science" \cite[2]{Braude1979}. Why is replication so important for scientific inquiry? 

\paragraph{Standard to judge scientific claims} 
Replication, or at the least reproducibility, opens claims to scrutiny; allowing us to keep what works and discard what doesn't. Science, according to the American Physical Society, ``is the systematic enterprise of gathering knowledge \ldots organizing and condensing that knowledge into testable laws and theories." The ``ultimate standard" for evaluating these scientific claims is whether or not the claims can be replicated \cite[]{Peng2011,Kelly2006}. Research findings cannot even really be considered ``genuine contribution[s] to human knowledge" until they have been verified through replication \cite[38]{Stodden2009}. Replication ``requires the complete and open exchange of data, procedures, and materials". Scientific conclusions that are not replicable should be abandoned or modified ``when confronted with more complete or reliable \ldots evidence".\footnote{See the American Physical Society's website at \url{http://www.aps.org/policy/statements/99_6.cfm}. See also \cite{Fomel2009}.} 

\paragraph{Avoiding effort duplication \& encouraging cumulative knowledge development} 
Not only is reproducibility crucial for evaluating scientific claims, it can also help enable the cumulative growth of future scientific knowledge \cite[]{Kelly2006,King1995}. Reproducible research cuts down on the amount of time scientists have to spend gathering data or developing procedures that have already been collected or figured out. Because researchers do not have to discover on their own things that have already been done, they can more quickly apply these data and procedures to building on established findings and developing new knowledge.

\subsection{For You}

Working to make your research reproducible does require extra upfront effort. For example, you need to put effort into learning the tools of reproducible research by doing things such as reading this book. But beyond the clear benefits for science, why should you make this effort? Using research reproducible tools can make your research process more effective and (hopefully) ultimately easier.

\paragraph{Better work habits}
Making a project reproducible from the start encourages you to use better work habits. It can spur you to more effectively plan and organize your research. It should push you to bring you data and source code up to a higher level of quality than you might if you ``thought `no one was looking'" \cite[386]{Donoho2010}. This forces you to root out errors--a ubiquitous part of computational research-earlier in the research process \cite[385]{Donoho2010}. Clear documentation also makes it easier to find errors.\footnote{Of course, it's important to keep in mind that reproducibility is ``neither necessary nor sufficient to prevent mistakes" \cite[]{Stodden2009b}.}

Reproducible research needs to be stored so that other researchers can actually access the data and source code. By taking steps to make you research accessible for others you are also making it easier for you to find your data and methods when you revise your work or begin new projects. You are avoiding personal effort duplication; allowing you to cumulatively build on your own work more effectively.

\paragraph{Better teamwork}
The steps you take to make sure an independent researcher can figure out what you have done also make it easier for your collaborators to understand your work and build on it. This applies not only to current collaborators, but also future collaborators. Bringing new members of a research team up to speed on a cumulatively growing research project is faster if they can easily understand what has been done already \cite[386]{Donoho2010}. 

\paragraph{Changes are easier}
A third person may or may not actually reproduce your research even if you make it easy for them to do so. But, {\emph{you will almost certainly reproduce parts or even all of your own research}}. Almost no actual research process is completely linear. You almost never gather data, run analyses, and present you results without going backwards to add variables, make changes to your statistical models, create new graphs, alter results tables in light of new findings, and so on. You will probably try to make these changes long after you last worked on the project and long since you remembered the details of how you did it. Whether your changes are because of journal reviewers' and conference participants' comments or you discover that new and better data has been made available since beginning the project, designing your research to be reproducible from the start makes it much easier to change things later on.  

Dynamically reproducible documents in particular can make changes much easier. Changes made to one part of a research project have a way of cascading through the other parts. For example, adding a new variable to a largely completed analysis requires gathering new data and merging it with existing data sets. If you used data imputation or matching methods you may need to rerun these models. You then have to update your main statistical analyses, and recreate the tables and graphs you used to present the results. Adding a new variable essentially forces you to reproduce large portions of your research. If when you started the project you used tools that make it easier for others to reproduce your research, you also made it easier to reproduce the work yourself. You will have taken steps to have a ``better relationship with [your] future [self]" \cite[]{Bowers2011}.

\paragraph{Higher research impact}
Reproducible research is more likely to be useful for other researchers than non-reproducible research. Useful research is cited more frequently \cite[]{Donoho2002,Piwowar2007,Vandewalle2012}. Research that is fully reproducible contains more information, i.e. more reasons to use and cite it, than presentation documents merely showing findings. Independent researchers may use the reproducible the data or code to look at other, often unanticipated, questions. When they use your work for a new purpose they will (should) cite your work. Because of this, Vandewalle et al. even argue that ``the goal of reproducible research is to have more impact with our research"  \citeyearpar[1253]{Vandewalle2007}.

A reason researchers often avoid making their research fully reproducible is that they are afraid other people will use their data and code to compete with them. I'll let Donoho et al. address this one:

\begin{quote}
    True. But competition means that strangers will read your papers, try to learn from them, cite them, and try to do even better. If you prefer obscurity, why are you publishing? \citeyearpar[16]{Donoho2009}
\end{quote}

\section{Who should read this book?}

This book is intended primarily for researchers who want to use a systematic workflow that encourages reproducibility and the practical state-of-the-art computer tools to put it into practice. This includes professional researchers, upper-level undergraduate, and graduate students working on computational data-driven projects. Hopefully, editors at academic publishers will also find the book useful for improving their ability to evaluate and edit reproducible research. 

The more researchers that use the tools of reproducibility the better. So I include enough information in the book for people who have very limited experience with these tools, including limited experience with R, LaTeX, and Markdown. They will be able to start incorporating these tools into their workflow right away. The book will also be helpful for people who already have general experience using technologies such as the R and LaTeX, but would like to know how to tie them together for reproducible research. 

\subsection{Academic Researchers}
Hopefully so far in this chapter I've convinced you that reproducible research has benefits for you as a member of the scientific community and personally as a computational researcher. This book is intended to be a practical guide for how to actually make your research reproducible. Even if you already use tools such as R and LaTeX you may not be leveraging their full potential. This book will teach you useful ways to get the most out of them as part of a reproducible research workflow.

\subsection{Students}
Upper-level undergraduate and graduate students conducting original computational research should make their research reproducible for the same reasons that professional researchers should. Forcing yourself to clearly document the steps you took will also encourage you to think more clearly about what you are doing and reinforce what you are learning. It will hopefully give you a greater appreciation of research accountability and integrity early in your careers \cite[183]{Barr2012,Ball2012}.

Even if you don't have extensive experience with computer languages, this book will teach you specific habits and tools that you can use throughout your student research and hopefully your careers. Learning these things earlier will save you considerable time and effort later.

\subsection{Instructors}
When instructors incorporate the tools of reproducible research into their assignments they not only build students' understanding of research best practice, but are also better able to evaluate and provide meaningful feedback on students' work \cite[183]{Ball2012}. This book provides a resource that you can use with students to put reproducibility into practice.

If you are teaching computational courses, you may also benefit from making your lecture material dynamically reproducible. Your slides will be easier to update for the same reasons that it is easier to update research.  Making the methods you used to create the material available to students will give them more information. Clearly documenting how you created lecture material can also pass information on to future instructors. 

\subsection{Editors}
Beyond a lack of reproducible research skills among researchers, an impediment to actually creating reproducible research is a lack of infrastructure to publish it \cite[]{Peng2011}. Hopefully, this book will be useful for editors at academic publishers who want to be better at evaluating reproducible research, editing it, and developing systems to make it more widely available. The journal {\emph{Biostatistics}} is a good example of a publication that is encouraging (actually requiring) reproducible research. From 2009 the journal  has had an editor for reproducibility that ensures replication files are available and that results can be replicated using these files \cite[]{Peng2009}. The more editors there are with the skills to work with reproducible research the more likely it is that researchers will do it.

\subsection{Private sector researchers}

Researchers in the private sector may or may not want to make their work easily reproducible outside of their organization. However, that does not mean that significant benefits cannot be gained from using the methods of reproducible research. First, even if public reproducibility is ruled out to guard proprietary information,\footnote{There are ways to enable some public reproducibility without revealing confidential information. See \cite{Vandewalle2007} for a discussion of one approach.} making your research reproducible to members of your organization can spread valuable information about how analyses were done and data was collected. This will help build your organization's knowledge and avoid effort duplication. Just as a lack of reproducibility hinders the spread of information in the scientific community, it can hinder it inside of a private organization. 

Also, the tools of reproducible research covered in this book enable you to create professional standardized reports that can be easily updated or changed when new information is available. In particular, you will learn how to create batch reports based on quantitative data.

%%%%%%%%%%%%%%%%% The Tools of Reproducible Research %%%%%%%%%%%%%%%

\section{The Tools of Reproducible Research}

This book will teach you the tools you need to make your research highly reproducible. Reproducible research involves two broad sets of tools. The first is a {\bf{reproducible research     environment}}\index{reproducible research        environment} that includes the statistical tools you need to run your analyses as well as ``the ability to automatically track the provenance of data, analyses, and results and to package them (or pointers to persistant versions of them) for redistribution". The second set of tools is a {\bf{reproducible research publisher}}\index{reproducible research publisher}, which prepares dynamic documents for presenting results and is easily linked to the reproducible research environment \cite[415]{Mesirov2010}.

In this book we will focus on learning how to use the widely available and highly flexible reproducible research environment--R/RStudio \cite[]{RLanguage,RStudioCite}. R/RStudio can be linked to numerous reproducible research publishers such as LaTeX and Markdown with Yihui Xie's {\emph{knitr}} package \citeyearpar{R-knitr}. The main tools covered in this book include:

\begin{itemize}
    \item {\bf{R}}: a programming language primarily for statistics and graphics. It can also be used for data gathering and creating presentation documents.
    
    \item {\bf{{\emph{knitr}}}}: an R package for literate programming\index{literate programming}, i.e. it allows you to combine your statistical analysis and the presentation of the results into one document. It works with R and a number of other languages such as Bash, Python, and Ruby.
    
    \item {\bf{Markup languages}}: instructions for how to format a presentation document. In this book we cover LaTeX and Markdown.  
    
    \item {\bf{RStudio}}: an integrated developer environment (IDE)\index{integrated developer  environment} for R that tightly integrates R, {\emph{knitr}}, and markup languages.
    
    \item {\bf{Cloud storage \& versioning}}: Services such as Dropbox and Github that can store data, code, and presentation files, save previous versions of these files, and make this information widely available.
    
    \item {\bf{Unix-like shell programs}}\index{Unix-like shell program}: These tools are useful for working with large research projects.\footnote{In this book I cover the Bash shell for Linux\index{Linux} and Mac as well as Windows PowerShell\index{Windows PowerShell.}} They also allow us to use command line tools including Pandoc, a program for converting documents from one markup language to another.
\end{itemize}

%%%%%%%%%%%%%%%%%%% Why use R, knitr, and RStudio for reproducible research? %%%%%%%%%%%%%%

\section{Why use R, knitr, and RStudio for reproducible research?}

\paragraph{Why R?}
Why use a statistical programming language like R for reproducible research? R has a very active development community that is constantly expanding what it is capable of. As we will see in this book this enables researchers across a wide range of disciplines to gather data and run statistical analyses. Using the {\emph{knitr}} package, you can connect your R-based analysess to presentation documents created with markup languages\index{markup language} such as LaTeX and Markdown. This allows you to dynamically and reproducibly present results in articles, slideshows, and webpages. 

The way you interact with R has benefits for reproducible research. In general you interact with R (or any other programming and markup language) by explicitly writing down your steps as source code\index{source code}. This promotes reproducibility more than your typical interactions with Graphical User Interface (GUI)\index{Graphical User Interface}\index{GUI} programs like SPSS\footnote{I know you can write scripts in statistical programs like SPSS, but doing so is not encouraged by the program's interface and you often have to learn multiple languages just to write scripts that run analyses, create graphics, and deal with matrices.} and Microsoft Word\index{Microsoft Word}. When you write R code and embed it in presentation documents created using markup languages you are forced to explicitly state the steps you took to do your research. When you do research by clicking through drop down menus in GUI programs, your steps are lost, or at least documenting them requires considerable extra effort. Also it is generally more difficult to dynamically embed your analysis in presentation documents created by GUI word processing programs in a way that will be accessible to other researchers both now and in the future. I'll come back to these points in Chapter \ref{GettingStartedRR}.

\paragraph{Why knitr?}

Literate programming\index{literate programming} is a crucial part of reproducible quantitative research.\footnote{Donald Knuth\index{Donald Knuth} coined the term literate programming in the 1970s to refer to a source file that could be both run by a computer and ``woven" with a formatted presentation document \cite[]{Knuth1992}.} Being able to directly link your analyses, your results, and the code you used to produce the results makes tracing your steps much easier. There are many different literate programming tools for a number of different programming languages. Previously, one of the most common tools for researchers using R and the LaTeX markup language was Sweave \cite[]{Leisch2002}.\index{Sweave} The package I am going to focus on in this book is newer and is called {\emph{knitr}}\index{knitr}. Why are we going to use {\emph{knitr}} in this book and not Sweave or some other tool?

The simple answer is that {\emph{knitr}} has the same capabilities as Sweave plus more. It can work with markup languages other than LaTeX\footnote{It works with LaTeX, Markdown, HTML, and reStructuredText\index{reStructuredText}. We cover the first two in this book.} and can even work with programming languages other than R. It highlights R code\index{syntax highlighting} in presentation documents making it easier for your readers to follow.\footnote{Syntax highlighting uses different colors and fonts to distinguish different types of text. For example in the PDF version of this book R commands are highlighted in \hlfunctioncall{maroon}, while character strings are in \hlstring{lavender}.} It gives you better control over the inclusion of graphics and can cache code chunks--save the output for later\index{cache code chunks}. It has the ability to understand Sweave-like syntax, so it will be easy to convert backwards to Sweave if you want to. You also have the choice to use much simpler and more straightforward syntax with {\emph{knitr}}. 

\paragraph{Why RStudio?}

\index{RStudio}Why use the RStudio integrated development environment for reproducible research? R by itself has the capabilities necessary to gather data, analyse it, and, with a little help from {\emph{knitr}} and markup languages, present results in a way that is highly reproducible. RStudio allows you to do all of these things, but simplifies many of them and allows you to navigate through them more easily. It is a happy medium between R's text-based interface and a pure GUI. 

Not only does RStudio do many of the things that R can do but more easily, it is also a very good stand alone editor for writing documents with LaTeX and Markdown. For LaTeX documents it can, for example, insert frequently used commands like \texttt{\textbackslash{}section\{\}} for numbered sections (see Chapter \ref{LatexChapter}).\footnote{If you are more comfortable with a what-you-see-is-what-you-get (WYSIWYG)\index{WYSIWYG} word processer like Microsoft Word, you might be interested in exploring Lyx\index{Lyx}. It is a WYSIWYG-like LaTeX editor that works with {\emph{knitr}}. It doesn't work with the other markup langages covered in this book. For more information see: \url{http://www.lyx.org/}. I give some brief information on using Lyx with \emph{knitr} in Chapter 3's Appendix.}  There are many LaTeX editors available, both open source and paid. But RStudio is currently the best program for creating reproducible LaTeX and Markdown documents. It has full syntax highlighting\index{syntax highlighting}. It's syntax highlighting can even distinguish between R code and markup commands in the same document. It can spell check LaTeX \& Markdown documents. It handles {\emph{knitr}} code chunks\index{code chunk} beautifully (see Chapter \ref{GettingStartedRKnitr}). Basically, RStudio makes it easy to create and navigate through complex documents. 

Finally, RStudio not only has tight integration with various markup languages, it also has capabilities for using other tools such as C++, CSS, JavaScript, and a few other programming languages. It is closely integrated with the version control programs Git\index{Git} and SVN\index{SVN}. Both of these programs allow you to keep track of the changes you make to your documents (see Chapter \ref{Storing}). This is important for reproducible research since version control programs can document many of your research steps. 

\subsection{Installing the Software}\label{InstallR}

Before you read this book you should install the software. All of the software programs covered in this book are open source and can be easily downloaded for free. They are available for Windows\index{Windows}, Mac\index{Mac}, and Unix-like operating systems\index{Unix}. They should run well on most modern computers. 

You should install R before installing RStudio. You can download the programs from the following websites:

\begin{itemize}
    \item {\bf{R}}: \url{http://www.r-project.org/},
    \item {\bf{RStudio}}: \url{http://www.rstudio.com/ide/download/}.
\end{itemize}

\noindent The download webpages for these programs have comprehensive information on how to install them, so please refer to those pages for more information.

After installing R and RStudio you will probably also want to install a number of user-written packages that are covered in this book. To install all of these user-written packages, please see page \pageref{ReqPackages}.

\paragraph{Installing markup languages}

If you are planning to create LaTeX documents you need to install a LaTeX distribution\index{LaTeX distribution}. They are available for Windows, Mac, and Unix. They can be found at: \url{http://www.latex-project.org/ftp.html}. Please refer to that site for more installation information.

If you want to create Markdown documents you can separately install the {\emph{markdown}} package\index{markdown package} in R. You can do this the same way that you install any package in R, with the {\tt{install.packages}} command.\footnote{The exact command is: {\tt{install.packages("markdown")}}.} 

%%%%%%%%%%%%%% Book Overview %%%%%%%%%%%%%%

\section{Book overview}

The purpose of this book is to give you the tools that you will need to do reproducible research with R and RStudio. 

This book describes a workflow for reproducible research primarily using R and RStudio. It is designed to give you the necessary tools to use this workflow for your own research. It is not designed to be a complete introduction to R, RStudio, {\emph{knitr}}, GitHub, or any other program that is a part of this workflow. Instead it shows you how these tools can fit together to make yourß research more reproducible. To get the most out of these individual programs I will along the way point you to other resources that cover these programs in more detail.

To that end, I can recommend a number of resources that cover more of the nitty-gritty:

\begin{itemize}
    \item Michael J. Crawley's encyclopaedic R book, appropriately titled, \textbf{The R Book} published by Wiley.
    
    \item Norman Matloff's tour through the programming language aspects of  R called \textbf{The Art of R Programming: A Tour of Statistical Design Software} published by No Starch Press.
    
    \item For an excellent introduction to the command line\index{command line} in Linux and Mac, though with pretty clear implications for Windows users if they are running PowerShell (see Chapter 2) see William E. Shotts Jr.'s book \textbf{The Linux Command Line: A Complete Introduction} also published by No Starch Press.
    
    \item The RStudio website (\url{http://www.rstudio.com/ide/docs/}) has a
  number of useful tutorials on how to use {\emph{knitr}} with LaTeX and Markdown.
\end{itemize}

That being said, my goal is for this book to be {\emph{self-sufficient}}. A reader without a detailed understanding of these programs will be able to understand and use the commands and procedures I cover in this book. While learning how to use R and the other programs I personally often encountered illustrative examples that included commands, variables, and other things that were not well explained in the texts that I was reading. This caused me to waste many hours trying to figure out, for example, what the \texttt{\$} is used for (preview: it's the component selector). I hope to save you from this wasted time by either providing a brief explanation of these possibly frustrating and mysterious conventions and/or pointing you in the direction of a good explanation.

\subsection{How to read this book}

This book gives you a workflow. It has a beginning, middle, and end. So, unlike a reference book it can and should be read linearly as it takes you through an empirical research processes from an empty folder to a completed set of documents that reproducibly showcase your findings.

That being said, readers with more experience using tools like R or LaTeX may want to skip over the nitty-gritty parts of the book that describe how to manipulate data frames or compile LaTeX documents into PDFs. Please feel free to skip these sections.

If you are experienced with R in particular you may want to skip over the first section of Chapter \ref{GettingStartedRKnitr}: Getting Started with R/RStudio. But don't skip over the whole chapter. The later parts contains important information on the {\emph{knitr}} package. 

\subsection{How this book was written}

This book practices what it preaches. It can be reproduced. I wrote the book using the programs and methods that I describe. Full documentation and source files can be found at the book's GitHub\index{GitHub} repository. Feel free to read and even use (within reason and with attribution, of course) the book's source code. You can find it at: \url{https://github.com/christophergandrud/Rep-Res-Book}. This is especially useful if you want to know how to do something in the book that I don't directly cover in the text.

\todo[inline]{During the writing of this book, the repository is private and cannot be accessed publicly.} 

\subsection{Contents overview}

The book is broken into four parts. The first part (chapters \ref{GettingStartedRR},  \ref{GettingStartedRKnitr}, and \ref{DirectoriesChapter}}) gives an overview of the reproducible research workflow as well as the general computer skills that you'll need to use this workflow. Each of the next three parts of the book guide you through the specific skills you will need for each part of the reproducible research process. The second part of the book (chapters \ref{Storing}, \ref{DataGather}, and \ref{DataClean}) covers the data gathering and file storage process. The third part (chapters \ref{StatsModel}, \ref{TablesChapter}, and \ref{FiguresChapter}) teaches you how to dynamically incorporate your statistical analysis, results figures and tables into your presentation documents. The final part (chapters \ref{LatexChapter}, \ref{LargeDocs}, and \ref{MarkdownChapter}) covers how to create reproducible presentation documents including LaTeX articles, books, slideshows and batch reports as well as Markdown webpages and slideshows.

