% Chapter Chapter 13 For Reproducible Research in R and RStudio
% Christopher Gandrud
% Created: 16/07/2012 05:45:03 pm CEST
% Updated: 13 October 2012




\chapter{Presenting on the Web and Beyond with Markdown/HTML}\label{MarkdownChapter}

\todo[inline]{This chapter is incomplete.}

\section{The Basics}

\subsection{Headings}

Headings in Markdown are extremely simple. To create a line in the style of the topmost heading--maybe a title--just place one hash mark (\texttt{\#}) at the beginning of the line. The second tier heading just gets two hashes (\texttt{\#\#}) and so on. You can also put the hash mark(s) at the end of the heading, but this is not necessary.

\subsection{Footnotes and bibliographies with MultiMarkdown}

\subsection{Math}

\subsection{Drawing figures with CSS}

\section{Presentations with \texttt{Slidify}}

\section{Simple webpages}

\subsection{RPubs}

\subsection{Hosting webpages with Dropbox}

\section{Reproducible websites}

\subsection{Blogging with Tumblr}

\subsection{Jekyll-Bootstrap and GitHub}

see \url{http://jfisher-usgs.github.com/r/2012/07/03/knitr-jekyll/}

\subsection{Jekyll and Github Pages}

\section{Using Markdown for non-HTML output with Pandoc}

Markdown syntax is very simple. So simple, you may be tempted to write many or all of your presentation documents in Markdown. This presents the obvious problem of how to convert your markdown documents to other markup languages if, for example, you want to create a LaTeX formatted PDF. As we saw in the previous chapter, Pandoc can help solve this problem. Pandoc is a command line program that can convert files written in Markdown, HTML, LaTeX, and a number of other markup languages\footnote{See the Pandoc website for more details: \url{http://johnmacfarlane.net/pandoc/}} to any of the other formats. 

%% Fill In Example with Fake Documents.

  
