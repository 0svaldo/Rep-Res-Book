% Chapter Chapter 7 For Reproducible Research in R and RStudio
% Christopher Gandrud
% Created: 16/07/2012 05:45:03 pm CEST
% Updated: 2 September 2012




\chapter{Preparing Data for Analysis}\label{DataClean}

Once we have gathered the raw data that we want to include in our statistical analyses we generally need to clean it and merge it into a single data files CAWELY QOUTE ABOUT HOW IT CAN BE BAD TO USE DATA FROM DIFFERENT DATA FRAMES.This chapter covers some of the basics of how to clean data files and merge them using R. 

The two main suggestions for cleaning and merging data are to:

\begin{itemize}
    \item always versions of the original--non-cleaned--data in as raw a state as possible,
    \item again document everything.
\end{itemize}

It's a good idea to keep data your original data in as raw a version as possible because it makes reconstructing the steps you took to create your data set easier. Also, while cleaning and merging your data you may transform it in an unintended way, for example, accidentally deleting some observations that you had intended to keep. Having the raw data makes it easy to go back and correct your mistake. Documenting everything also helps you achieve these two goals. Also it makes updating the data set easier if, for example, new data becomes available. MAYBE EXPLAIN MORE.

If you are very familiar with data transformations in R you may want to skip onto the next chapter. 

\section{Cleaning data for merging}

\section{Sorting data}

\section{Merging data sets}

\section {Subsetting data}
