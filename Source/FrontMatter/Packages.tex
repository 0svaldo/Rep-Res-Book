


\chapter*{Required R Packages} \label{ReqPackages}

In this book I discuss how to use a number of user-written R packages for reproducible research. Many of these packages are not included in the default R installation. They need to be installed separately. To install all of the user-written packages discussed in this book type the following code:

\begin{knitrout}
\definecolor{shadecolor}{rgb}{0.969, 0.969, 0.969}\color{fgcolor}\begin{kframe}
\begin{alltt}
\hlfunctioncall{install.packages}(\hlstring{"animation"},
                \hlstring{"apsrtable"}, 
                \hlstring{"countrycode"},
                \hlstring{"devtools"}, 
                \hlstring{"formatR"},
                \hlstring{"gdata"},
                \hlstring{"ggplot2"}, 
                \hlstring{"googleVis"},
                \hlstring{"httr"},
                \hlstring{"httr"},
                \hlstring{"knitr"}, 
                \hlstring{"knitcitations"}, 
                \hlstring{"markdown"}, 
                \hlstring{"openair"},
                \hlstring{"plyr"}, 
                \hlstring{"quantmod"},
                \hlstring{"reshape"},
                \hlstring{"reshape2"},
                \hlstring{"RCurl"},
                \hlstring{"rjson"},
                \hlstring{"RJSONIO"},
                \hlstring{"texreg"},
                \hlstring{"tools"},
                \hlstring{"treebase"},
                \hlstring{"twitteR"},
                \hlstring{"WDI"},    
                \hlstring{"XML"},                 
                \hlstring{"xtable"}, 
                \hlstring{"Zelig"})
\end{alltt}
\end{kframe}
\end{knitrout}


\noindent Once you enter this code, you may be asked to select a CRAN ``mirror"\index{mirrors, CRAN} to download the packages from.\footnote{CRAN stands for the Comprehensive R Network.} Simply select the mirror closest to you.

 Ramnath Vaidyanathan's \emph{Slidify}\index{Slidify} package \citeyearpar{R-slidify} for creating R Markdown/HTML slideshows (see Chapter \ref{MarkdownChapter}) is not currently on CRAN. It can be downloaded directly from GitHub\index{GitHub}. To do this first load the \emph{devtools} package \cite[]{R-devtools}.\footnote{} Then download \emph{Slidify}. Here is the complete code:

\begin{knitrout}
\definecolor{shadecolor}{rgb}{0.969, 0.969, 0.969}\color{fgcolor}\begin{kframe}
\begin{alltt}
\hlcomment{# Load devtools}
\hlfunctioncall{library}(devtools)

\hlcomment{# Install Slidify and ancillary libraries}
\hlfunctioncall{install_github}(\hlstring{"slidify"}, \hlstring{"ramnathv"})
\hlfunctioncall{install_github}(\hlstring{"slidifyLibraries"}, \hlstring{"ramnathv"})
\end{alltt}
\end{kframe}
\end{knitrout}


\noindent For more details see the \emph{Slidify} website: \url{http://ramnathv.github.com/slidify/start.html#}.

If you are Windows you will also need to install Rtools \cite[]{Rtools}.\index{Rtools} You can download Rtools from: \url{http://cran.r-project.org/bin/windows/Rtools/}.\label{RtoolsDownload}

\todo[inline]{Fix write\_bib issue}



