\documentclass{article}

\usepackage{tikz}

\begin{document}

%%%%%% Example 

\tikzstyle{DirBox} = [draw=black, rectangle, minimum width=5em, thick]
  
\begin{tikzpicture}
  % Root Directory
  \node (root) at (5, 10) [DirBox]{Root};
  
  % Project Directory
  \node (project) at (5, 8.5) [DirBox]{Project};
  
  % Main Project sub-directories
  \node (data) at (2, 7) [DirBox]{Data};
  \node (analysis) at (5, 7) [DirBox]{Analysis};
  \node (presentation) at (8, 7) [DirBox]{Presentation};
  
  % Data subdirectories/files
  \node (dataGatherSource) at (-0.5, 6) [DirBox]{GatherSource};  
  \node (dataFiles) at (2, 6) [DirBox]{DataFiles};
  
  \node (dataMake) at (-0.5, 5.25) {{\small{DataMakeFile.R}}};
  
  \node(dataGather) at (-0.5, 4.5) [DirBox]{IndvDataGather};
  \node (dataGather1) at (-0.5, 3.5) {{\small{DataGather1.R}}};
  
  % Analysis subdirectores/files
  
  % Presentation subdirectories/files
  
  % Connect boxes
  \draw (root) -- (project);
  \draw (project) -| (data);
  \draw (project) -- (analysis);
  \draw (project) -| (presentation);
  
  %% Data connect boxes
  \draw (data) -| (dataGatherSource);
  \draw (data) -- (dataFiles);
  \draw (dataGatherSource) -- (dataMake);
  
\end{tikzpicture}



\end{document}