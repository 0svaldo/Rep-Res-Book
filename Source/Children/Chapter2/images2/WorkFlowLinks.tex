%%%%%%%%%%%%%%
% Reproducible research workflow + link commands
% Christopher Gandrud
% Updated 28 September 2012
%%%%%%%%%%%%%%

\documentclass{article}\usepackage{graphicx, color}
%% maxwidth is the original width if it is less than linewidth
%% otherwise use linewidth (to make sure the graphics do not exceed the margin)
\makeatletter
\def\maxwidth{ %
  \ifdim\Gin@nat@width>\linewidth
    \linewidth
  \else
    \Gin@nat@width
  \fi
}
\makeatother

\IfFileExists{upquote.sty}{\usepackage{upquote}}{}
\definecolor{fgcolor}{rgb}{0.2, 0.2, 0.2}
\newcommand{\hlnumber}[1]{\textcolor[rgb]{0,0,0}{#1}}%
\newcommand{\hlfunctioncall}[1]{\textcolor[rgb]{0.501960784313725,0,0.329411764705882}{\textbf{#1}}}%
\newcommand{\hlstring}[1]{\textcolor[rgb]{0.6,0.6,1}{#1}}%
\newcommand{\hlkeyword}[1]{\textcolor[rgb]{0,0,0}{\textbf{#1}}}%
\newcommand{\hlargument}[1]{\textcolor[rgb]{0.690196078431373,0.250980392156863,0.0196078431372549}{#1}}%
\newcommand{\hlcomment}[1]{\textcolor[rgb]{0.180392156862745,0.6,0.341176470588235}{#1}}%
\newcommand{\hlroxygencomment}[1]{\textcolor[rgb]{0.43921568627451,0.47843137254902,0.701960784313725}{#1}}%
\newcommand{\hlformalargs}[1]{\textcolor[rgb]{0.690196078431373,0.250980392156863,0.0196078431372549}{#1}}%
\newcommand{\hleqformalargs}[1]{\textcolor[rgb]{0.690196078431373,0.250980392156863,0.0196078431372549}{#1}}%
\newcommand{\hlassignement}[1]{\textcolor[rgb]{0,0,0}{\textbf{#1}}}%
\newcommand{\hlpackage}[1]{\textcolor[rgb]{0.588235294117647,0.709803921568627,0.145098039215686}{#1}}%
\newcommand{\hlslot}[1]{\textit{#1}}%
\newcommand{\hlsymbol}[1]{\textcolor[rgb]{0,0,0}{#1}}%
\newcommand{\hlprompt}[1]{\textcolor[rgb]{0.2,0.2,0.2}{#1}}%

\usepackage{framed}
\makeatletter
\newenvironment{kframe}{%
 \def\at@end@of@kframe{}%
 \ifinner\ifhmode%
  \def\at@end@of@kframe{\end{minipage}}%
  \begin{minipage}{\columnwidth}%
 \fi\fi%
 \def\FrameCommand##1{\hskip\@totalleftmargin \hskip-\fboxsep
 \colorbox{shadecolor}{##1}\hskip-\fboxsep
     % There is no \\@totalrightmargin, so:
     \hskip-\linewidth \hskip-\@totalleftmargin \hskip\columnwidth}%
 \MakeFramed {\advance\hsize-\width
   \@totalleftmargin\z@ \linewidth\hsize
   \@setminipage}}%
 {\par\unskip\endMakeFramed%
 \at@end@of@kframe}
\makeatother

\definecolor{shadecolor}{rgb}{.97, .97, .97}
\definecolor{messagecolor}{rgb}{0, 0, 0}
\definecolor{warningcolor}{rgb}{1, 0, 1}
\definecolor{errorcolor}{rgb}{1, 0, 0}
\newenvironment{knitrout}{}{} % an empty environment to be redefined in TeX

\usepackage{alltt}

\usepackage{pdflscape}
\usepackage[usenames,dvipsnames]{xcolor}
\usepackage{tikz}
\usetikzlibrary{decorations.pathmorphing}
\usetikzlibrary{shapes,arrows}

\begin{document}

% Define colors for figure
%% Color palette (GnBU) chosen using ColorBrewer 2.0
%% See: http://colorbrewer2.org/
\definecolor{Blue}{HTML}{7BCCC4}
\definecolor{LiteBlue}{HTML}{A8DDB5}

% Set node styles
%% Workflow stage nodes
\tikzstyle{Stage} = [draw=Blue, 
                     fill=Blue, 
                     rectangle, 
                     text width=7em, 
                     inner sep=0.5cm, 
                     font=\large]

% Raw Data nodes
\tikzstyle{RawData} = [draw=LiteBlue, 
                       fill=LiteBlue, 
                       decorate,
                       decoration={random steps,
                                   segment length=2pt,
                                   amplitude=2pt},
                       inner sep=0.25cm, 
                       font=\normalsize]
% Link command nodes       
\tikzstyle{every node} = [draw=none, 
                          %anchor=west,
                          %text=
                          font=\small]
                          
% Line Style
\tikzstyle{line} = [draw, 
                    -latex',
                    very thick]

% Begin tikz picture
\begin{landscape}
\begin{tikzpicture}
  % Nodes
    % Workflow stage nodes
    \node (DataGather) at (1, 5) [Stage]{Data Gather};
    \node (Analysis) at (6, 5) [Stage]{Analysis};
    \node (Presentation1) at (11, 9) [Stage]{Article \& \\ Slideshow \\ Presentations};
    \node (Presentation2) at (11, 1) [Stage]{Website \\ Presentations};
    
    % Raw Data Nodes
    \node (Data1) at (-3, 8) [RawData]{Raw Data};
    \node (Data2) at (-3, 5) [RawData]{Raw Data};
    \node (Data3) at (-3, 2) [RawData]{Raw Data};

    % Lines
    \path [line] (Data1) -- (DataGather);
    \path [line] (Data2) -- (DataGather);
    \path [line] (Data3) -- (DataGather);
    \path [line] (DataGather) -- (Analysis);
    \path [line] (Analysis) -- (Presentation1);
    \path [line] (Analysis) -- (Presentation2);'
    
    % Link command nodes

  
  
  
\end{tikzpicture}
\end{landscape}

\end{document}
