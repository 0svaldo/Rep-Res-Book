% Chapter Chapter 7 For Reproducible Research in R and RStudio
% Christopher Gandrud
% Created: 16/07/2012 05:45:03 pm CEST
% Updated: 12 January 2012




\chapter{Preparing Data for Analysis}\label{DataClean}

Once we have gathered the raw data that we want to include in our statistical analyses we generally need to clean so that it can be merged it into a single data file. In this chapter we will learn how to create the data gather and merging files we saw last chapter. The chapter also includes information on recoding and transforming variables. This is important for merging data, but will be very useful information in later chapters as well. If you are very familiar with data transformations in R you may want to skip onto the next chapter. 

\section{Cleaning data for merging}

In order to successfully merge two or more data frames we need to make sure that they are in the same format. Let's look at some of the important formatting issues and how to reformat your data frames so that they can be easily merged.

\subsection{Get a handle on your data}

Before doing anything to your data it is a good idea to take a look at it and see what needs to be done. Surprisingly, just taking a little time to look at your data will help you avoid many error messages and much frustration. 

To get a sense of your data you could of course just type a data frame object's name into the R console. This will print the entire data frame. For data frames with more than a few variables and observations. We have already seen a number of commands that are useful for seeing parts of your data. As we saw in Chapter \ref{GettingStartedRKnitr}, \texttt{names}\index{R command!names} command shows you the variable names of a data frame object. The \texttt{head}\index{R command!head} command shows the first few observations in a data frame and \texttt{tail}\index{tail} shows the last few.

The \texttt{summary} command\index{R command!summary} is especially helpful for seeing not only basic descriptive statistics for all of the variables in a data frame, but also the variables' types. For example, let's us the \emph{FertConsumpData} object we created in Chapter \ref{DataGather}:

{\small
\begin{knitrout}
\definecolor{shadecolor}{rgb}{0.969, 0.969, 0.969}\color{fgcolor}\begin{kframe}
\begin{alltt}
\hlcomment{# Summarize FertConsumpData data frame object}
\hlfunctioncall{summary}(FertConsumpData)
\end{alltt}
\begin{verbatim}
## Length  Class   Mode 
##      0   NULL   NULL
\end{verbatim}
\end{kframe}
\end{knitrout}

}

\noindent We can immediately see that the variables \textbf{iso2c} are character strings. Because \emph{summary} is able to calculate means, medians, and so on for \textbf{AG.CON.FERT.ZS} and \textbf{year} we know they are numeric. You can of course run \emph{summary} on a particular variable by using the component selector (\verb|$|):

\begin{knitrout}
\definecolor{shadecolor}{rgb}{0.969, 0.969, 0.969}\color{fgcolor}\begin{kframe}
\begin{alltt}
\hlcomment{# Summarize the methane emissions variable from FertConsumpData}
\hlfunctioncall{summary}(FertConsumpData$AG.CON.FERT.ZS)
\end{alltt}
\begin{verbatim}
## Length  Class   Mode 
##      0   NULL   NULL
\end{verbatim}
\end{kframe}
\end{knitrout}


\noindent We'll come back to why knowing this type of information is important for merging and data analysis later in this Chapter.

You can view a portion of a data frame object with the \texttt{View} command.\index{R command!view} This will open a new window that lets you see a selection of the data frame. If you are using RStudio, you can click on the data frame in the \emph{Workspace} tab and you will get something that look similar. Note that neither of these viewers are interactive in that you can't use them to manipulate the data. They are only data viewers. To be see similar windows that you can interactively edit use the \texttt{fix} command in the same way that you use \texttt{view}. This can be useful for small edits, but remember that the edits are not reproducible.

\subsection{Reshaping Data}

Obviously it is usually a good if the data sets kept in data frame type objects. See Chapter \ref{GettingStartedRKnitr} (page \pageref{data.frame}) for how to convert objects into data frames with the \texttt{data.frame} command. Not only do data sets (generally) need to be stored in data frame objects they also need to follow the same layout before they can be merged. Most R statistical analysis tools assume that your data is in ``long'' format\index{long formatted data} (as we also did in Chapter \ref{GettingStartedRKnitr}). This usually means that data frame columns are variables and rows are specific observations (see Table \ref{ExampleLong}).

\begin{table}[h!]
    \caption{Long Formatted Data Example}
    \label{ExampleLong}
    \begin{tabular}{l c}
        \\[0.15cm]
        \hline
        Subject & Variable1 \\
        \hline \\[0.1cm]
        Subject1 & \\[0.25cm]
        Subject2 & \\[0.25cm]
        Subject3 & \\[0.25cm]
        \ldots & \\[0.25cm]
        \hline
    \end{tabular}
\end{table}

\noindent In this chapter we will mostly use examples of time-series cross-sectional data (TSCS)\index{time-series cross-sectional}\index{TSCS} that we want to have in long-format. Long formatted TSCS has is simply a data frame where rows identify observations of a particular subject at three points in time (see Table \ref{ExampleTSCSLong})

 \begin{table}[h!]
    \caption{Long Formatted Time-series Cross-sectional Data Example}
    \label{ExampleTSCSLong}
    \begin{tabular}{l c c}
        \\[0.15cm]
        \hline
        Subject & Time & Variable1 \\
        \hline \\[0.1cm]
        Subject1 & 1 & \\[0.25cm]
        Subject1 & 2 & \\[0.25cm]
        Subject1 & 3 & \\[0.25cm]
        Subject2 & 1 & \\[0.25cm]
        Subject2 & 2 & \\[0.25cm]
        Subject2 & 3 & \\[0.25cm]
        \ldots & & \\[0.25cm]
        \hline
    \end{tabular}
\end{table}

\noindent In this chapter our TSCS data is specifically going to be countries that are observed in multiple years.

\noindent If one of your data sets is not in this format then you will need to reshape\index{reshape data} it. Some data sets are in ``wide'' format;\index{wide formatted data} where one of the columns in long formatted data is widened to cover multiple columns. This can be confusing without an example. Table \ref{ExampleWide} shows how Table \ref{ExampleTSCSLong} looks when we widen the time variable.

\begin{table}[h!]
    \caption{Wide Formatted Data Example}
    \label{ExampleWide}
    \begin{tabular}{l c c c}
        \\[0.15cm]
        \hline 
        Subject & Time1 & Time2 & Time3 \\
        \hline \\[0.1cm]
        Subject1 & & & \\[0.25cm]
        Subject2 & & & \\[0.25cm]
        \ldots & & & \\[0.25cm]
        \hline
    \end{tabular}
\end{table}

Reshaping data is often the cause of much confusion and frustration. Though probably never easy, there are a number of useful R functions for changing data from wide format to long and vice versa. These include the matrix transpose command (\textbf{t})\footnote{See this example by Rob Kabacoff: \url{http://www.statmethods.net/management/reshape.html}. Note also that because the matrix transpose function is denoted with simply as \texttt{t}, you should not give any object the name \emph{t}.}\index{matrix transpose} and the \textbf{reshape}\index{R command!reshape} command, both in loaded in R by default.  Another very helpful package is \emph{reshape2} \citep{R-reshape2}.\index{reshape2} This provides more general tools for reshaping data and is worth investing some time in learning well. In this section we will cover some of \emph{reshape2}'s basic commands and use them to reshape TSCS data frame from wide to long format. We will also encounter this package in more detail in Chapter \ref{FiguresChapter} when we want to transform data so that it can be graphed.

Let's imagine that the fertilizer consumption data we previously downloaded from the World Bank is in wide rather than long format and is in a data frame objected called \emph{WideFert}. It looks like this:\footnote{Please see the Appendix (page \pageref{WideAppendix}) for the code I used to reshape the data.}















































