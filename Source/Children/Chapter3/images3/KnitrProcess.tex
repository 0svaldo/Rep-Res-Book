%%%%%%%%%%%%%%
% The knitr process
% Christopher Gandrud
% Updated 8 January 2012
%%%%%%%%%%%%%%

% Define colors for figure
%% Color palette (GnBU) chosen using ColorBrewer 2.0
%% See: http://colorbrewer2.org/
%% Not used in the print version
\definecolor{Blue}{HTML}{7BCCC4}
\definecolor{LiteBlue}{HTML}{A8DDB5}
\definecolor{DarkBlue}{HTML}{08589E}

\definecolor{GrayLine}{HTML}{BDBDBD}

% Set node styles
%% Workflow stage nodes
\tikzstyle{Docs} = [draw=Blue, 
                     rectangle, 
                     %text width=7em, 
                     inner sep=0.3cm, 
                     font=\small]

% Begin tikz picture
\begin{tikzpicture}

	\node(knit) at (2, 1.75) {\emph{\textbf{Knit}}};
	\node(compile) at (6, 1.75) {\emph{\textbf{Compile}}};

	% Document nodes
	\node (knitable) at (0, 0) [Docs, text width= 6em]{Knitable Document \\ (Markup + Analysis Code)};
	\node (Markup) at (4, 0) [Docs, text width= 6em]{Markup Only Document};
	\node (Presentation) at (8, 0) [Docs, text width = 6em]{Presentation Document};
	};

	% LaTeX Example
	\node(LaTeX) at (0, -2.5) {\textbf{LaTeX Example}};
	\node (Rnw) at (0, -3.5) [Docs, text width= 6em]{\emph{Paper.Rnw}};
	\node (tex) at (4, -3.5) [Docs, text width= 6em]{\emph{Paper.tex}};
	\node (pdf) at (8, -3.5) [Docs, text width = 6em]{\emph{Paper.pdf}};
	};

	% Markdown Example
	\node(Markdown) at (0, -5) {\textbf{Markdown Example}};
	\node (Rmd) at (0, -6) [Docs, text width= 6em]{\emph{Website.Rnw}};
	\node (md) at (4, -6) [Docs, text width= 6em]{\emph{Website.md}};
	\node (html) at (8, -6) [Docs, text width = 6em]{\emph{Website.html}};
	};

	% Lines
	\draw [->, very thick] (knitable) -- (Markup);
	\draw [->, very thick] (Markup) -- (Presentation);

	\draw [->, very thick] (Rnw) -- (tex);
	\draw [->, very thick] (tex) -- (pdf);

	\draw [->, very thick] (Rmd) -- (md);
	\draw [->, very thick] (md) -- (html);

\end{tikzpicture}