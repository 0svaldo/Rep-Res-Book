% Chapter Chapter 11 For Reproducible Research in R and RStudio
% Christopher Gandrud
% Created: 16/07/2012 05:45:03 pm CEST
% Updated: 29 September 2012




\chapter{Presenting with LaTeX}\label{LatexChapter}

\section{The Basics}

All commands in LaTeX start with a \texttt{\textbackslash{}}

\subsection{Editors}

As I mentioned earlier, RStudio is an fully functional LaTeX editor as well as an integrated development environment for R. Of course it is oriented towards combining R and LaTeX. If you want to create a new LaTeX document you can click {\tt{File}} \rightarrow {\tt{New}} \rightarrow {\tt{R\; Sweave}}. 

Remember from Chapter \ref{GettingStartedRKnitr} that R Sweave\index{R Sweave} files are basically LaTeX files that can include {\emph{knitr}} code chunks. You can compile R Sweave files like regular LaTeX files in RStudio even if they do not have code chunks. If you use another program to compile them you might need to change the file extension from {\tt{.Rnw}} to {\tt{.tex}}.

\subsection{The header \& the body}

All LaTeX documents require a header\index{LaTeX header}. The header goes before the body of the document and specifies what type of presentation document you are creating--an article, a book, a slideshow, and so on. LaTeX refers to these as classes\index{LaTeX class}. You also can specify what style it should be formatted in and load any extra packages you may want to use to help you format your document.\footnote{The command to load a package in LaTeX is \texttt{\textbackslash{}usepackage}. For example, if you include \texttt{\textbackslash{}usepackage\{url\}} in the header of your document you will be able to specify URL links in the body with the command \texttt{\textbackslash{}url\{SOMEURL\}}.}

The header is followed by the body of your document. You tell LaTeX where the body\index{LaTeX begin document} of your document starts by typing \texttt{\textbackslash{}begin\{document\}}. The very last line of you document is usually \texttt{\textbackslash{}end\{document\}}, indicating that your document has ended. When you open a new R Sweave file in RStudio it creates an article class document with a very simple header and body like this:















