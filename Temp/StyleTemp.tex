I use the following conventions throughout this book to format computer
code:

\begin{itemize}
\item
  \textbf{Abstract Variables}
\end{itemize}
Abstract variables, i.e.~variables that do not represent specific
objects in an example, are in \texttt{ALL CAPS TYPWRITER TEXT}.

\begin{itemize}
\item
  \textbf{Clickable Buttons}
\end{itemize}
Clickable Buttons are in \texttt{typewriter text}.

\begin{itemize}
\item
  \textbf{Code}
\end{itemize}
All code is in \texttt{typewriter text}.

\begin{itemize}
\item
  \textbf{Filenames and Directories}
\end{itemize}
Filenames and directories more generally are printed in \emph{italics}.
Camelback is used for file and directory names.

\begin{itemize}
\item
  \textbf{Individual variable values}
\end{itemize}
Individual variable values mentioned in the text are in \textbf{bold}.

\begin{itemize}
\item
  \textbf{Objects}
\end{itemize}
Objects are printed in \emph{italics}. Camelback (e.g.~CamelBack) is
used for object names.

\begin{itemize}
\item
  \textbf{Columns}
\end{itemize}
Columns are printed in \emph{italics}

\begin{itemize}
\item
  \textbf{Packages}
\end{itemize}
\textbf{R} packages are printed in \emph{italics}.

\begin{itemize}
\item
  \textbf{Windows}
\end{itemize}
Open windows are written in \textbf{bold} text.

\begin{itemize}
\item
  \textbf{Variable Names}
\end{itemize}
Variable names are printed in \emph{italics}. Camelback is used for
individual variable names.
