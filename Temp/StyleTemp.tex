I use the following conventions throughout this book to format computer
code and actions:

\begin{itemize}
\item
  Abstract Variables
\end{itemize}
Abstract variables, i.e.~variables that do not represent specific
objects in an example, are in \texttt{ALL CAPS TYPWRITER TEXT}.

\begin{itemize}
\item
  Clickable Buttons
\end{itemize}
Clickable Buttons are in \texttt{typewriter text}.

\begin{itemize}
\item
  Code
\end{itemize}
All code is in \texttt{typewriter text}.

\begin{itemize}
\item
  Filenames and Directories
\end{itemize}
Filenames and directories more generally are printed in \emph{italics}.

Camelback is used for file and directory names.

\begin{itemize}
\item
  Individual variable values
\end{itemize}
Individual variable values mentioned in the text are in \textbf{bold}.

\begin{itemize}
\item
  Objects
\end{itemize}
Objects are printed in \emph{italics}.

Camelback is used for objects.

\begin{itemize}
\item
  Object Classes
\end{itemize}
Object classes are in \texttt{typewriter text}.

\begin{itemize}
\item
  Packages
\end{itemize}
\textbf{R} packages are printed in \emph{italics}.

\begin{itemize}
\item
  Windows
\end{itemize}
Open windows are written in \textbf{bold} text.

\begin{itemize}
\item
  Variable Names
\end{itemize}
Variable names are printed in \emph{italics}.

Camelback (e.g.~CamelBack) is used for individual variable names.
